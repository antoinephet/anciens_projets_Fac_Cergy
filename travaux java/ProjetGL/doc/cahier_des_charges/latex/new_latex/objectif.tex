
\section{Objectif}
\label{sec:Objectif}

\paragraph{} Nous avons présenté l'objectif du projet dans la section \ref{sec:introduction}. Dans cette section, nous présentons la spécification de notre logiciel réalisé. Ceci correspond principalement au cahier des charges.

\subsection{Description brève du logiciel}
\label{sec:spec1}
\paragraph{}L’objectif du projet consiste à développer un jeu ou une application basé sur un système de type «multi-agents réactifs» mettant en place comme éléments : un gardien, des intrus et des éléments du décor ou de l’interface graphique (eau, arbre, mur…).

\subsection{Le but du jeu}
\label{sec:spec2}
\paragraph{}Cela consiste à l’utilisateur à prendre en main un gardien qui a pour but d’attraper les intrus présents dans un environnement donné (ici un parc). L’idée est divertir l’utilisateur afin qu’il soit immergé dans le cadre du divertissement.

\subsection{Liste des fonctionnalités}
\label{sec:spec3}
\paragraph{}Cette application permettra de réaliser les fonctionnalités suivantes
\paragraph{Présenter au joueur une interface d’entrée dans le jeu}: cette interface expliquera à l’utilisateur la tâche qu’il devra accomplir.
\paragraph{- Mettre en évidence des éléments importants de la scène en les surlignant}l’utilisateur sera donc guidé implicitement.
\paragraph{- Offrir une simplicité d'utilisation }: le jeu devra être facile à utiliser et à manipuler.
\paragraph{- Offrir une rapidité de mise en œuvre}: le jeu doit permettre à des joueurs de mesurer leurs réflexes mentaux afin qu’il puisse prendre des décisions le plus rapidement possible lors de la partie du jeu.
\paragraph{- Assurer des temps d'exécution faibles }: lors du lancement du logiciel, les temps de chargement des programmes doivent être rapides.
\paragraph{- Assurer des temps d'interactions joueur / machine faibles }: durant l’exécution du logiciel, l’utilisateur ne doit sentir aucune latence, que ce soit avec le clavier ou la souris. Les déplacements du joueur doivent être fluides.

\subsection{Contraintes}
\label{sec:spec4}
\paragraph{}- Problème de repérage des intrus par le gardien lié a son champ de vision limité.\\
- Détermination de la position des intrus\\
- Présence de multiple intrus dans l'espace.\\
- Déplacements et simulation des intrus.\\

\newpage 

\section{Livraisons attendues}
\label{sec:Livraisons}
\paragraphe{}À la fin de notre projet, nous pensons rendre les documents suivants :
\begin{itemize}
  \item des programmes en langage java comportant une interface graphique  
  \item l’exécutable jar des programmes
  \item manuel d'utilisation  du logiciel
  \item un rapport
  \item un cahier des charges final
	\item un planning (en cours)
  \item javadoc
\end{itemize}

